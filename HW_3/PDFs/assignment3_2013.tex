\documentclass{report}
\usepackage{color}
\begin{document}

\title{\textbf{PHYS-4810 Project Proposal:} 
\\Double Pendulum on a Cart}
\author{A.L. Phillips II\\
  Department of Physics, Astronomy, and Applied Physics,\\
  Rensselaer Polytechnic Institute\\
  \texttt{philla3@rpi.edu}}
 \date{19 April 2013}
 \renewcommand{\chaptername}{Assignment}
 \setcounter {chapter}{2}
\maketitle
\section*{The Physics}

\begin{enumerate}

\item Consider the problem of planetary motion around 

\item Write a \textsc{C++} code, using Runge-Kutta at fourth order 

	\begin{itemize}

\item First discuss the

\item Then, turn on the Jupiter-Earth gravitational

\item In both cases, use realistic numbers for the 

\end{itemize}

\end{enumerate}




\section*{Algorithmic Considerations}

\begin{enumerate}

\item Consider the problem of planetary motion around 

\item Write a \textsc{C++} code, using Runge-Kutta at fourth order 

	\begin{itemize}

\item First discuss the

\item Then, turn on the Jupiter-Earth gravitational

\item In both cases, use realistic numbers for the 

\end{itemize}

\end{enumerate}


\section*{Obtaining a Solution}

\begin{enumerate}

\item Establish tolerance criterion of the model (Reaction times, maximum angular displacement, etc.)

\item Free body diagram to obtain system of equations 

	\begin{itemize}

\item First discuss the

\item Then, turn on the Jupiter-Earth gravitational

\item In both cases, use realistic numbers for the 

\end{itemize}

\end{enumerate}






\section*{The Goal}

\begin{enumerate}

\item To establish a thorough appreciation of the model. \\Display an understanding of the models:

	\begin{itemize}

\item Theoretical derivations.

\item Physical limitations

\item Usefulness in 'real-world' applications.

\end{itemize}

\end{enumerate}



Make sure your report is self-contained with sufficient details and
clear plots. Feel free to add listings of your codes (or use
pseudo-code). 
\end{document}
