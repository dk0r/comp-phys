\documentclass{report}
\usepackage{graphicx}
\usepackage{wrapfig}
\begin{document}

\title{\textbf{Homework 4:} Partial Differential Equations (PDEs): relaxation and leapfrog methods}
\author{C.F. Gauss\\
  Department of Physics, Astronomy, and Applied Physics,\\
  Rensselaer Polytechnic Institute\\
  \texttt{gauss@rpi.edu}}
 \date{04/01/2012}
\maketitle
\section{Assignment}
This homework assignment focuses on numerical methods to solve PDE equations. 
\begin{enumerate}

\item Elliptical PDE: relaxation methods. We have solved a number of cases involving Poisson or
Laplace equations during the lectures. Here, I want you to write a
program to realize this relaxation, and provide a detailed report of your findings.
You code will include :
\begin{enumerate}
\item Simple relaxation method (Jacobi)
\item Gauss-Seidel Method
\item Over- and Under- relaxation method. 
\end{enumerate}
Your work will include the following items.
\begin{enumerate}
\item Calculate the potential in the system depicted on the figure.  Study the effect of grid size (test a few different values for separation and thickness). 
\item Plot equipotential lines and compute the electric field and discuss the physics. 
\item Find the optimal value of the over-relaxation parameter (make a plot of the number of iterations needed to reach a set precision
versus relaxation parameter). Make a convincing case when presenting
your findings. Here you will use a 128 $\times$ 128 grid.
\end{enumerate}
\item Hyperbolic or Parabolic equations: select a problem of your choice where the leapfrog method can be used. Implement the leapfrog method, discuss the physics of your problem and comment on the numerical results and numerical stability. You are allowed to use a problem discussed in class, provided you consider different boundary/initial conditions. 
\item (optional) This part of the assignment is for extra credits (up to 2). As you
  will soon realize, the speed at which the relaxation method
  converges depends critically on the quality of starting guess (you could test
  that for the first problem above, by artificially setting the
  starting potentials at various values, clearly out of the range of
  the solution; for instance, putting all the unknown values at -100V
  should be worse than setting them all between 0 and 10). A good
  approach to accelerate convergence is by using a multigrid approach
  (MGA). In the MGA, a coarse grid is first used to perform the
  relaxation. The converged result is then interpolated into a finer
  grid, and relaxation is then performed using that new grid.
  
  You can use interpolation of your choice to move from one grid to
  the next (you can simply copy the value at one point of the old grid
  to the new four points defined by the new grid, or you can use
  linear or cubic interpolation). Here, I suggest you use a 64x64 grid as a coarse grid
and the 128x128 grid as a fine grid.  

For those students who want to meet a challenge, I suggest to
write a more general code that will use any number of grids (for
example, up to 7 grids for the $128 \times 128$ finest grid), by
successively interpolate into a finer and finer grid For this second
part, the student will choose a Laplace/Poisson problem of their
choice. It is important to show and discuss the improvements of the
MGA compared to the usual one (i.e. plot the execution time as
function of grid size).
\end{enumerate}
\begin{figure}[p]
  \begin{center}
      \includegraphics[width=.6\columnwidth]{h3_fig1.pdf}
  \end{center}
\caption{Setup for question 1(a). The central cross is set at 10V and the boundary is grounders (0V).}
\end{figure}


Make sure your report is self-contained with sufficient details and
clear plots. Feel free to add listings of your codes (or use
pseudo-code). 

\textbf{Important Note:} One point will be subtracted for each missing or incomplete figure caption or missing axis labels.

\end{document}
