\documentclass{report}
\usepackage{color}
\begin{document}

\title{\textbf{Homework 3:} Ordinary Differential Equations (ODEs): physics at work}
\author{C.F. Gauss\\
  Department of Physics, Astronomy, and Applied Physics,\\
  Rensselaer Polytechnic Institute\\
  \texttt{gauss@rpi.edu}}
 \date{\today}
 \renewcommand{\chaptername}{Assignment}
 \setcounter {chapter}{2}
\maketitle
\section*{Assignment}
\color{red} Due date: 03/22/2013 \color{black}
\begin{enumerate}
\item Consider the problem of planetary motion around the Sun, as
  discussed during lecture 8. 
\begin{enumerate}
\item Write a \textsc{C++} code, using Runge-Kutta at fourth order
  (\textsc{RK4}) to describe the motion of Earth and Jupiter around
  the Sun. You will use the \textsc{RK4} library provided in class. 
\begin{itemize}
\item First discuss the trajectories when Jupiter and Earth do not
  interact. (Hint: you will have 2 bodies, 2 spatial variables per
  body and no cross-terms)
\item Then, turn on the Jupiter-Earth gravitational
  interaction. (Hint: same as before but with a cross-term between
  Jupiter and Earth)
\item In both cases, use realistic numbers for the orbits (size of the
  orbit, velocity at aphelion or perihelion, etc...). Verify Kepler's
  laws and compute periodicity. 
\item Repeat your calculations, by artificially increasing the mass of
  Jupiter by a factor of 1,000. Discuss the effect on the orbits. Make
  sure you make clear plots of the orbits.
\end{itemize}
Note: it is a good idea to use astronomical units.
\item Study the precession of the perihelion of Mercury due to general
  relativity corrections to the $1/r^2$ classical gravitational
  law. Use the following equation of the modified force:
$$ F_G=\frac{GM_SM_M}{r^2}(1+\frac{\alpha}{r^2})$$
where $M_M$ and $M_S$ correspond to the mass of Mercury and the Sun,
respectively. Use $\alpha\sim10^{-8}$ as a small coefficient
accounting to relativity corrections.
\begin{itemize}
\item Plot the trajectory of Mercury for a given $\alpha=0.01
  AU^2$. Draw the line between the Sun and the closest approach of
  Mercury for a few trajectories. The change in direction indicates the
  change of orientation of the perihelion. 
\item Plot the orbit's orientation change with time for $\alpha=0.0008 AU^2$.
\end{itemize}
In all these calculations, do not consider the effect of the planets
on the Sun (Sun is static); choose initial conditions properly, test
your step size $h$ carefully.
\end{enumerate}

\item Select a problem of your choice from any physics class (or book)
  where ODEs cannot be solved analytically. Present the physics of the
  problem carefully and make a case for the solution you obtained, in
  light of the conditions you selected.
\end{enumerate}


Make sure your report is self-contained with sufficient details and
clear plots. Feel free to add listings of your codes (or use
pseudo-code). 
\end{document}
