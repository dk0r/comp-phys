\documentclass{report}
\begin{document}

\title{\textbf{Homework 2:} Random Numbers, Monte Carlo, and Multidimensional Integration}
\author{C.F. Gauss\\
  Department of Physics, Astronomy, and Applied Physics,\\
  Rensselaer Polytechnic Institute\\
  \texttt{gauss@rpi.edu}}
 \date{02/25/2011}
 \renewcommand{\chaptername}{Assignment}
 \setcounter {chapter}{1}
\maketitle
\chapter{Random Numbers}
\section{Assignment}
\begin{enumerate}
\item Use your computer's default pseudo-RNG to
  generate a sequence $\{x_i\}_{i=1,N}$ of random numbers in the range between 0.0 and 1.0 (choose an
  appropriate value for $N$, such as 10,000 or more).
\begin{enumerate}
\item Plot $x_i$ as a function of sequence number $i$. Comment on the
  plot. Does it \textit{look} random and uniform?
\item Plot $x_i$ as a function of $x_{i+k}$ for a small $k$ of your choice. What can you learn from this plot?
\item Compute the $k^{th}$ moment of your distribution. Compare to the theoretical value and comment.
\item Calculate the auto-correlation between $x_i$ and $x_{i+k}$ for a few values of $k$. Comment and discuss.
\item Prepare a plot with \textit{bins} to assess the uniformity of the sequence. Use a reasonable number of bins (10, for instance). What does this plot tell you about the sequence?
\item Using the information in (a)-(e), comment on the quality of your sequence of random numbers.
\item Repeat with a sequence of numbers obtained from random.org.
\end{enumerate} 
\item Choose one of the two following problems:
\begin{itemize}
\item \textbf{Use random numbers to calculate the integral in five dimensions:}
\begin{equation}
\int_0^1 dx_1 \int_0^1 dx_2 \int_0^1 dx_3 \int_0^1 dx_4 \int_0^1 dx_5  (x_1+x_2+x_3+x_4+x_5)^3
\end{equation}
\begin{enumerate}
\item Compare your result with the analytical result.
\item Repeat the calculation for a varying number of random numbers. 
\item Study how the error behaves as you change the number of
  points. Plot this error and find if you can detect a $\sim
  1/\sqrt{N}$ behavior of the error, where $N$ is the number of points.
\item Compare the order of magnitude of your error with (numerically obtained):
\begin{equation}
\sigma^2=<f^2>-<f>^2
\end{equation}
where 
\begin{equation}
f=(x_1+x_2+x_3+x_4+x_5)^2
\end{equation}
\end{enumerate}
\item \textbf{Random walk in 3D}
  \begin{enumerate}
\item Repeat the 2D work presented in class for an equivalent 3D random walk.
\item At each step, consider a constant step size (i.e. of length 1.0,
  thereby defining the unit of distance). This means you could use
  spherical coordinates with constant $r=1$, while randomly picking
  values for azimuthal angle $\theta \in [0,2\pi]$ and zenith angle
  $\phi\in[0,\pi]$. Feel free to use Cartesian coordinates if you
  prefer; but in that case make sure you normalize the displacement
  vector.
\item Vary the total number $N$ of steps and plot $\sqrt{<R^2>}$ as a
  function of $\sqrt{N}$. $<R^2>$ could be computed by averaging over,
  say, $n_{average}=16$ different random walks (choose a different
  value for $n_{average}$ if you want). Comment on the general
  behavior of the $\sqrt{<R^2>}$ versus $\sqrt{N}$ plot.
\item Plot a few trajectories and discuss their shape.
\end{enumerate}
\end{itemize}
\end{enumerate}

Make sure your report is self-contained with sufficient details and
clear plots. Feel free to add listings of your code (or use
pseudo-code). Collaborative work is allowed but each student will turn
in an individual report.

Below is a suggestion for presentation. You can adapt it to your personal vision of the problem.
\section{Algorithmic Considerations}
\section{Implementation}
\section{Results}
\section{Discussion}
\section{References (if any)}

\end{document}
