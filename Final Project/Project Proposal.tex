\documentclass{report}
\usepackage{color}
\begin{document}

\title{\textbf{PHYS-4810 Project Proposal:} 
\\A Double Pendulum on a Cart}
\author{A.L. Phillips II\\
  Department of Physics, Astronomy, and Applied Physics,\\
  Rensselaer Polytechnic Institute\\
  \texttt{philla3@rpi.edu}}
 \date{19 April 2013}
 \renewcommand{\chaptername}{Assignment}
 \setcounter {chapter}{2}
\maketitle
\section*{The Physics}

\begin{enumerate}

\item Establish a force diagram and extract the component equations of the forces acting on each center of mass.

\item Manipulate force equations via substitutions/simplifications in order to obtain the coupled second order differential equations of motion. 

\end{itemize}

\end{enumerate}


\\
\\

\section*{Algorithmic Considerations}

\begin{enumerate}

\item The above system of second order ODE's must be transformed into canonical form in order to be solved numerically.   

\item Remain aware of rounding errors and their affect upon solutions.

\end{enumerate}

\\
\\

\section*{Obtaining a Solution}

\begin{enumerate}

\item Establish tolerance criterion and assumptions for the model \\(Reaction times, maximum angular displacement, etc.)

\item Implement Runge-Kutta 4 to solve the system.


\end{enumerate}


\\
\\



\section*{The Goal}

\begin{enumerate}

\item To program a dynamic solution which manipulates a cart 
\\into establishing and maintaining balance in a double pendulum

\item Establish a thorough appreciation of the model by displaying an understanding of the:

	\begin{itemize}

\item Theoretical derivations

\item Physical limitations

\item Usefulness in 'real-world' applications

\end{itemize}

\item Produce an animation of the model

\end{enumerate}



\end{document}
